\documentclass[10pt,twocolumn,letterpaper]{article}

\usepackage{iccv}
\usepackage{times}
\usepackage{epsfig}
\usepackage{graphicx}
\usepackage{amsmath}
\usepackage{amssymb}

% Include other packages here, before hyperref.

% If you comment hyperref and then uncomment it, you should delete
% egpaper.aux before re-running latex.  (Or just hit 'q' on the first latex
% run, let it finish, and you should be clear).
\usepackage[breaklinks=true,bookmarks=false]{hyperref}

\iccvfinalcopy % *** Uncomment this line for the final submission

\def\iccvPaperID{****} % *** Enter the ICCV Paper ID here
\def\httilde{\mbox{\tt\raisebox{-.5ex}{\symbol{126}}}}

% Pages are numbered in submission mode, and unnumbered in camera-ready
%\ificcvfinal\pagestyle{empty}\fi
\setcounter{page}{1}
\begin{document}

%%%%%%%%% TITLE
\title
{
    Classification of Histology and Pathology Images Using Convolutional Neural
    Networks%
}

\author{%
    Luong Nguyen\\
    University of Pittsburgh\\
    Biomedical Building Tower 3\\
    Fifth Ave\\
    Pittsburgh, PA 15026\\
    % {\tt\small luongn@andrew.cmu.edu}
    \and
    Dan Spagnolo\\
    University of Pittsburgh\\
    Biomedical Building Tower 3\\
    Fifth Ave\\
    Pittsburgh, PA 15026\\
    % {\tt\small luongn@andrew.cmu.edu}
    \and
    Jakob Bauer\\
    Carnegie Mellon University\\
    5000 Forbes Avenue\\
    Pittsburgh, PA 15213\\
    % {\tt\small jsbauer@andrew.cmu.edu}
}

\maketitle
%\thispagestyle{empty}


%%%%%%%%% ABSTRACT
% \begin{abstract}
%    The ABSTRACT is to be in fully-justified italicized text, at the top
%    of the left-hand column, below the author and affiliation
%    information. Use the word ``Abstract'' as the title, in 12-point
%    Times, boldface type, centered relative to the column, initially
%    capitalized. The abstract is to be in 10-point, single-spaced type.
%    Leave two blank lines after the Abstract, then begin the main text.
%    Look at previous ICCV abstracts to get a feel for style and length.
% \end{abstract}

%%%%%%%%% BODY TEXT

\section{Introcuction}
\label{sec:Introcuction}

\section{Aims}
\label{sec:Aims}

\section{Dataset}
\label{sec:Dataset}

\section{Methods}
\label{sec:Methods}

\section{Expected Complications}
\label{sec:Expected Complications}

% \begin{figure}[h]
% \begin{center}
% \fbox{\rule{0pt}{2in} \rule{0.9\linewidth}{0pt}}
%    %\includegraphics[width=0.8\linewidth]{egfigure.eps}
% \end{center}
%    \caption{Example of caption.  It is set in Roman so that mathematics
%    (always set in Roman: $B \sin A = A \sin B$) may be included without an
%    ugly clash.}
% \label{fig:long}
% \label{fig:onecol}
% \end{figure}

\begin{figure*}
\begin{center}
\fbox{\rule{0pt}{2in} \rule{.9\linewidth}{0pt}}
\end{center}
   \caption{Example of a short caption, which should be centered.}
\label{fig:short}
\end{figure*}

% For this citation style, keep multiple citations in numerical (not
% chronological) order, so prefer \cite{Alpher03,Alpher02,Authors14} to
% \cite{Alpher02,Alpher03,Authors14}.
%
% When placing figures in \LaTeX, it's almost always best to use
% \verb+\includegraphics+, and to specify the  figure width as a multiple of
% the line width as in the example below
% {\small\begin{verbatim}
%    \usepackage[dvips]{graphicx} ...
%    \includegraphics[width=0.8\linewidth]
%                    {myfile.eps}
% \end{verbatim}
% }

%------------------------------------------------------------------------

\section{References}
\label{sec:References}

%-------------------------------------------------------------------------

{\small
\bibliographystyle{ieee}
\bibliography{egbib}
}

\end{document}
